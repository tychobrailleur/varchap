
\def\creerStyle{\@ifnextchar[\creer@style{\creer@style[]}}
\def\creer@style[#1]#2#3#4{%
\def\firstlettre##1##2\relax{##1}%
\edef\next{\def\noexpand\firstlettre{\firstlettre #2\relax}}%
\uppercase\expandafter{\next}%
\def\suitelettre##1##2\relax{##2}%
\expandafter\gdef\csname get\firstlettre\suitelettre#2\relax\endcsname{%
 #3%
 \nomStyle{#2}%
 #4\setkeys{v@rch@pkey}{#1}}
 \DeclareOption{#2}{\AtBeginDocument{%
\expandafter\csname get\firstlettre\suitelettre#2\relax\endcsname}}}


%%% Definition des cles
\define@key{v@rch@pkey}{coulNumChap}{\coulNumChap{\color{#1}}\relax}
\define@key{v@rch@pkey}{tailleNumChap}{\tailleNumChap{#1}\relax}
\define@key{v@rch@pkey}{fontNumChap}{\fontNumChap{#1}\relax}
\define@key{v@rch@pkey}{selectNumChap}[true]{\tailleNumChap{\selectfont}\relax}
\define@key{v@rch@pkey}{MEPNumChap}{\MEPNumChap{#1}\relax}

\define@key{v@rch@pkey}{arabeNumChap}[true]{\arabeNumChap}
\define@key{v@rch@pkey}{romainNumChap}[true]{\romainNumChap}

\define@key{v@rch@pkey}{chapitreNom}[true]{\chapitreNom\relax}
\define@key{v@rch@pkey}{pasChapitreNom}[true]{\pasChapitreNom\relax}

\define@key{v@rch@pkey}{coulTitreChap}{\coulTitreChap{\color{#1}}\relax}
\define@key{v@rch@pkey}{tailleTitreChap}{\tailleTitreChap{#1}\relax}
\define@key{v@rch@pkey}{fontTitreChap}{\fontTitreChap{#1}\relax}
\define@key{v@rch@pkey}{selectTitreChap}[true]{\tailleTitreChap{\selectfont}\relax}
\define@key{v@rch@pkey}{MEPTitreChap}{\MEPTitreChap{#1}\relax}

\define@key{v@rch@pkey}{MEPuni}{\MEPuni{#1}\relax}

\define@key{v@rch@pkey}{avantChapSkip}{\setlength{\avantchapitreskip}{#1}\relax}
\define@key{v@rch@pkey}{apresChapSkip}{\setlength{\apreschapitreskip}{#1}\relax}
\define@key{v@rch@pkey}{intersticeSkip}{\setlength{\intersticeskip}{#1}\relax}


\DeclareOption{msg}{\@stylemessagetrue}
\DeclareOption{christelle}{\AtBeginDocument{\getChristelle}}
\DeclareOption{ellen}{\AtBeginDocument{\getEllen}}
\DeclareOption{audrey}{\AtBeginDocument{\getAudrey}}
\DeclareOption{claudia}{\AtBeginDocument{\getClaudia}}
\DeclareOption{undine}{\AtBeginDocument{\getUndine{0pt}{2}}}
\DeclareOption{lisbeth}{\AtBeginDocument{\getLisbeth}}

\InputIfFileExists{varchap.cfg}\relax


%%%%%%%%%%%%%%


- Todo: remettre le code ci-dessus dans le vrai .dtx
- Remettre sylvia en option de ex-varchap.tex?
- Remettre � jour l'*image*
- Exemples tir�s de vrais livres.

